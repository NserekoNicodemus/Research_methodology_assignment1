\documentclass{article}
\begin{document}

\begin {small} \center \textbf{NSEREKO NICODEMUS    215023556     15/U/22184 }\end{small}\\

\begin{Large}\textbf{HOW TO CREATE A POTTERY ARTWORK.}\end{Large}\\
\section{Executive summary}{ 
One of the hobbies that can’t get off my mind is fine art. In this report is a step by step process taken as I create some of the beautiful art pieces from clay and this has turned out that it is pretty simple for anyone to come out with an art piece of interest provided he gives it time and uses the necessary requirements.}
{ 

The potter can form his product in one of many ways. Clay may be modeled by hand or with the assistance of a potter's wheel, may be jiggered using a tool that copies the form of a master model onto a production piece, may be poured into a mold and dried, or cut or stamped into squares or slabs. This brings us to the world of beauty that we all embrace knowingly or unknowingly in our daily lives.}
\section{Introduction}{
Pottery is clay that is modeled, dried, and fired, usually with a glaze or finish, into a vessel or decorative object. Pottery itself, has been in existence for a myriad of years. Pinch pots, made from balls of clay into which fingers or thumbs are inserted to make the opening, may have been the first pottery. Coil pots, formed from long coils of clay that are blended together, were not far behind.}
{

I personally as someone who grew up with the love for the art as a talent I discovered in me, came up with some simple steps on how anyone interested can come up with a simple pottery art work and make good use of it, maybe. Below are the steps I take to come out with a piece of pottery as one of my favorites.
}
\section{Raw materials and tools}{
The main material used is clay There are two types of clays, primary and secondary. Primary pure clay is found in the same place as the rock from which it is derived. Primary clay is heavy, dense, and pure. This secondary clay, a mixture of sediment, is finer and lighter than primary clay. The table below shows some of the materials and tools used.

\begin{tabular}{|c|ccccccc|}
\hline
$Materials$ &clay&glaze&slip \\ \hline
$Tools$ &ribs&texture tools&cutting tools& rolling pins&sponges&brush\\
 \hline
\end{tabular}

\section{The manufacturing process.}
\subsection{Mixing the clay:}{The powdered clay is moistened with water and mixed in a huge tank with a huge paddle. Multiple spindles mix and re-mix the clay, in order to evenly distribute water. The clay is then chopped into fine pieces to de-air it. The clay is then formed into cylinders that are now ready to be molded or formed.}
\subsection{Slip casting: }{ Pottery with delicate or intricate silhouette is often formed by slip casting. A pourable slip or slurry is poured into a two-part plaster mold, the excess is poured out, and the slip is permitted to stiffen and dry.}
\subsection{Glazing:}{The pieces may be entirely covered in one color of glaze by being run under a waterfall of glaze that completely coats each piece, or the pieces may be sprayed with glaze. Deep hollowware such as vases have to be flushed with glaze by hand to ensure that they are completely coated on the inside. }
\subsection{Firing:}{Kilns may be heated by gas, coal, or electricity. One large production potter uses tunnel kilns fired with natural gas. The pots remain in the kilns for about 5 hours. The kiln changes the glaze into a glass-like coating, which helps make the pot virtually impervious to liquid. Then final touches can be done by cleaning, grinding bottoms, polishing, sanding rough spots, which are done after the kiln has cooled and pots are removed. }
\section{Conclusion}{There are many other pottery techniques that I explore and they can be interchanged depending on what suites the desires. All that matters is the creativity and the interest of the mind. Art goes on developing with experience of nature, memories and the environment.}





\end{document}